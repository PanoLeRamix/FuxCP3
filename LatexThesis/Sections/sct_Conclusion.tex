\chapter*{Conclusion}
\addcontentsline{toc}{chapter}{Conclusion}
In conclusion, this thesis has made significant progress towards the development of a constraint programming based tool for creating music. However, it is important to note that the work presented here is still a work in progress, with several areas that require further exploration and refinement.

One of the key findings of this research is the recognition that a comprehensive formalization of musical rules is crucial for CP to be a relevant approach. The formalization of musical rules using discrete mathematics and constraints provides a solid foundation for generating musically correct solutions. However, it is essential to acknowledge that the process of formalizing all the intricate nuances of music is a challenging task. The use of more precise works could be useful to formalize the counterpoint even better.

The analysis of the generated counterpoint compositions based on Fux's rules has highlighted the need for additional constraints on melodic development, particularly in terms of long-range melodic relationships. While the tool successfully creates harmonically interdependent and melodically independent counterpoints, incorporating constraints that generate more interesting melodies would be a valuable direction. Also, from a technical point of view, the software architecture, performance, and quality of the solutions must be more taken into consideration in the future.

In addition, the successful experimentation outcome signifies that while the current solver represents an initial step, it holds great potential for more complex and advanced solvers in the future. These findings provide optimistic prospects for using similar solvers outside the domain of counterpoint. Indeed, the approach presented in this thesis can be extended to more complex musical styles. CP has the potential to be a powerful paradigm in computer-aided composition for a wide range of musical genres. The application of specific rules and constraints for different styles will open up new possibilities for composers and expand the creative potential of the tool.

In summary, while this thesis has laid a foundation for the development of a constraint programming based composition tool, there is still work to be done. Further research and development are needed to refine the formalization process, incorporate additional constraints on melodic development, and explore the application of CP in more complex musical styles. With continued efforts and advancements in these areas, we hope that constraint programming has the potential to revolutionize computer-aided composition and empower composers with new tools for musical creativity and expression.