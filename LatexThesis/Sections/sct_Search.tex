\chapter{Search method}
Three parts composition brings in way more possibilities than two parts composition. As Fux says is %todo insert quote
But more possibilities also mean a increased computation complexity. The search space has been extended a lot by adding a whole new set of variables, and the time taken for a solution to be found might to be too elevated if one does not think about optimizing the search. In addition to that, adding a third voice to a composition is not bringing many new constraints (which would help discarding some potential solutions faster), but instead comes with many preferences, which in constraint programming, are translated to costs.

\section{Changing the search algorithm}

To handle this, it was decided to switch from the previous Depth First Search algorithm to a more efficient Branch and Bound (BAB). This allows us to properly handle the costs, and to find faster solutions. Moreover, the BAB algorithm can also yields results than are not optimal, which is really valuable since finding the best solution overall might be time consuming. When launching the search for a solution, it is now possible to ask for the next solution (i.e. a better solution than the one that was found previously, and if none was found previously, then just any valid solution) or for the best solution. In the second case, the solver starts searching until it finds the best solution or until it is stopped, and yields a better solution each time it finds one.

Knowing that we are searching the solution whose cost must be the lowest possible raises the following question: how to compute the cost in order to best reflect the preferences expressed in \textit{Gradus ad Parnassum}?
The way of translating each preference into a corresponding cost have of course been formalised in the previous sections, but that's not the crux of the matter. The question we face here is: what is the best way to combine all these individual costs to obtain the most accurate result in terms of what Fux tries to transmit?

Three main ways of doing this have been identified:la première est la façon qui était utilisée dans la thèse de T. Wafflard, à savoir: il existait un coût total, $\tau$, qui était égal à la somme de tous les coûts individuels, $\mathcal{C}$. Il s'agissait ensuite de minimiser $\tau$. Chaque $\mathcal{C}_i$ étant lui même en général une somme de sous-coûts. Prenons par exemple le coût des motions, $\mathcal{C}_{motions} = \sum_j P_{costs}[j] $. Ce coût est la somme de tous les sous-coûts des motions (un par motion): par défaut, une motion contraire a un sous-coût de 0, une motion oblique a un sous-coût de 1 et une motion directe a un sous-coût de 2. Étant donné que tous les sous-coûts ont été définis sur la base d'une échelle (allant de 0 à 64$m$), il existe déjà une façon de comparer ces sous-coûts entre eux. Il nous est par exemple possible que le motion en $j=2$ est moins coûteux que la motion en $j=3$, car $P_{costs}[2] < P_{costs}[2]$. Par exemple, par défaut, une motion directe est deux fois pire qu'une motion oblique. Il n'est en revanche pas possible de comparer deux coûts entre eux: nous ne pourrions pas dire que le coût des motions est plus faible que le coût. En effet, ,  Nos leviers d'action, en choisissant ce type d'agencement des coûts, est de définir à chaque coût 

La seconde façon de deal with les coûts est de les agencer dans un array, puis d'effectuer une minimisation lexicographique. En d'autres termes, les coûts seraient alors rangés par ordre d'importance: du plus important au moins important. Cette méthode a beaucoup de sens si on réfléchit un peu aux phrases de \gap. Fux dit par exemple que les consonances parfaites peuvent être atteintes par une motion directe, s'il n'y a plus d'autre possibilité. Cela signifie que toutes choses étant égales par ailleurs, on préférera atteindre une consonance parfaite par motion oblique ou contraire, mais qu'entre une mauvaise solution (ne respectant presque aucune préférence) dans laquelle aucune consonance parfaite n'est atteinte par motion directe et une bonne solution (respectant presque toutes les préférences) dans laquelle une consonance parfaite est atteinte par motion directe, on choisira la bonne solution. 

\subsection{Heuristics}

\subsection{Costs} \label{costs}