\chapter{Introduction}
This thesis is a formalisation of three-part counterpoint based on the writings and rules of Fux. Its aim is to provide a mathematical set of rules and a computer environment capable of translating Fux's teachings into formal logic, and capable of implementing these logical rules in a concrete way to produce Fux-style counterpoint.


This thesis will therefore be divided into several parts: we will first immerse ourselves in \gap, Fux's central work, from which we will meticulously extract the rules laid down by its author. We will briefly discuss these rules to make them unambiguous, and then translate them into formal logic, so that each rule Fux had in mind when writing his work is mathematically recorded. On this basis, we will create a computer implementation using constraint programming. We will then look at how this implementation finds results, discussing the search algorithm and heuristics used. We then discuss the cost techniques used to obtain the best possible results. Finally, we will analyse the musical compositions produced by the tool created.

It is very important to know that this thesis is based on T. Wafflard's thesis "FuxCP: a constraint programming based tool formalizing Fux's musical theory of counterpoint" \cite{wafflard2023}. The present work takes up T. Wafflard's concepts and definitions, and could only be understood in its full depth by reading and fully understanding that work as well. A short summary of it is given in section \ref{section:thomas-in-a-nutshell}.



\section{A brief history of counterpoint: from the writings of Bach to algorithmic generation}
And before we get to the heart of the matter (which is the formalisation), let's take a look at Fux's theory of counterpoint, because that's the work that this work aims to formalise. Counterpoint is a compositional technique in which there are several musical lines (or voices) that are independent and distinct from one another but are balanced and sound beautiful \cite{CpSachs}. No voice is dominant over the others, and all are main voices, even though some may take a small precedence during part of the composition \cite{hess2016}.  


Counterpoint has been central to the work of many famous composers from different artistic movements, such as Bach in the Baroque, Mozart in the Classical and Beethoven in the Romantic, and has continued to arouse interest over the centuries, with the development of key texts on the subject, such as Schenker's Counterpoint \cite{schenker1906} or Jeppesen's Analysis \cite{jeppesen1960}. And while Bach was probably the master of counterpoint composition in his day \cite{yearsley2002}, the central and founding work in the teaching of counterpoint belongs to another great Baroque composer: the Austrian Fux and his treatise \gap. In it, this Baroque composer gives a detailed analysis of two-, three- and four-part writing of counterpoint, all told as a discourse between a master and his pupil. \gaps  was part of the species counterpoint movement, a way of conceiving counterpoint in five different types that could then be combined. This is the work on which this dissertation is based. 


For Fux, but also for many other authors, species counterpoint are governed by many diverse rules, and it is these rules that interest us in the present work. The rules are based on old concepts that go back to older styles \cite{crocker1962} and have been discussed by many authors.  Those concepts include, for example, the notions of opposite motion and consonance (which in turn can be either perfect or imperfect). These concepts and their application to counterpoint are particularly interesting because they allow us to consider the composition of counterpoint both in a 'vertical' way, in which we consider the harmony of the notes played together, and in a 'horizontal' way, in which we consider the melodic development of each of the parts individually, which provides the independence of the counterpoints from each other and their melodic beauty.

This is what makes it interesting to analyse from a constraint programming point of view. We'll come back to this later, but for now let's concentrate on Fux's music theory.

\subsection{Fux's theory of counterpoint}
As we have just said, Fux uses a set of rules to create 'the right counterpoint'. These rules can be divided into three categories: melodic rules, harmonic rules and rules of motion. We will examine them here, bearing in mind that the formalisation of all the rules for two-voice counterpoint can be found in T. Wafflard's thesis, and that the complete set of rules (in mathematical form) can be found in the appendix \ref{appendix:complete-set-of-rule} to this thesis.

\subsubsection{Melodic Rules}

Fux explains that there are rules that apply within parts (the horizontal rules) about the order of the interval between one note and the next: we find, for example, that a melody is more beautiful\footnote{Throughout this work we will speak of the "beauty of music". This beauty is highly subjective, and therefore reference will be made to the Fuxian concept of music to define whether a melody is beautiful or not. In other words, music will be considered beautiful if it conforms to the rules of Fux, and vice versa} if the intervals between its successive notes are small, that there is no chromatic succession, if the notes that follow each other are varied, and so on. These 'horizontal' rules are called 'melodic' rules because they concern only the melody. These rules apply within a given counterpoint.


\subsubsection{Harmonic Rules}


If there is a horizontal perspective to counterpoint, there is also, of course, a vertical perspective. This perspective is expressed in a harmonic relationship between the different voices. At each point in the composition, a series of rules (known as 'harmonic rules' because they concern harmony alone) apply. For example, there is the rule that in the first beats of each measure the interval between the voices must be a consonance \footnote{A third, a fifth, a sixth or an octave.}; imperfect consonances\footnote{Thirds and sixths.} are preferred to fifths, which are preferred to octaves; and the rule that the different voices must be different at each point in the composition. These rules apply between the counterpoints.

\subsubsection{Motion rules}
Finally, there is a third type of rule: movement rules. These rules are a hybrid of the two discussed above in that they consider not only vertical interaction, i.e. harmony, but also horizontal interaction, i.e. melody. They can therefore be seen as 'diagonal' rules that relate the unique melody of each counterpoint to its respective harmonies. These rules include, for example, the fact that we prefer voices that move in opposite directions (i.e. if one voice goes up, we want the other to go down), the rule that there should be no sequence of fifths or octaves between voices, or the rule that we should not arrive at a perfect interval by direct motion.

\subsection{Species counterpoint}\label{section:species-counterpoint}
When we talk about species counterpoint, we're talking about five categories of counterpoint. Each species is a separate concept, each with its own peculiarities. We'll go into more detail about the different species below. Let's first concentrate on how species counterpoint works. When composing counterpoint, the starting point is a fixed melodic line, the \cf, which is a basic melody composed entirely of whole notes. It is the basis of a composition when writing counterpoint. It is from this voice and in relation to it that the others are composed. It is important to note that once the composition is complete, the \cfs is no more or less important than the other voices, and has the same melodic independence as the other voices. It is therefore nothing more than the basis from which we begin to write.

Let's take a look at the five species:
\begin{enumerate}
    \item \textbf{First species}: Note against note \textendash{} the counterpoint is composed entirely of whole notes, and the composition is a sequence of harmonies sounding on the first beat between the counterpoint and the other voices.
    \item \textbf{Second species}: Half notes against whole notes \textendash{} the counterpoint is composed entirely of half notes, which introduce dissonant harmonies.
    \item \textbf{Third species}: Quarters against whole notes \textendash{} the counterpoint is made up entirely of quarter notes, which allow more different movements and more freedom in the composition.
    \item \textbf{Fourth species}: The ligature \textendash{} the counterpoint is delayed by two beats, creating syncopation. The notes are all round (i.e., tied minims, since they span between two measures).
    \item \textbf{Fifth species}: Florid counterpoint \textendash{} counterpoint is a mixture of all the other species and is the richest form of counterpoint. It allows great freedom of composition while respecting the rules of the other types.
\end{enumerate}

Although these types could technically be combined to form a three-part composition with two different types of counterpoint, Fux seems to prefer to write a 'special' counterpoint (i.e. second, third, fourth or fifth species) in one voice and only whole notes (i.e. a first species counterpoint) in another voice. He also sometimes says that it is possible and recommended to mix species, but does not do so extensively.

\subsection{Building a computer tool for writing species counterpoint}
We have just seen that counterpoint is an ancient style, and many generations of composers have worked on it (and still do). So it's only natural that, over the years, some computer specialists have thought about automating the writing of counterpoint. The first notorious example of this was Schottstaedt \cite{bill1984}, who in 1984 developed an expert system for writing counterpoint based on Fux's rules. Confronted with the technologies and algorithms of his time, he created an algorithm consisting of over 300 if-else clauses, with the drawback that these if-else clauses are far less efficient than what constraints are capable of today.

In 1997, a genetic programming and symbiosis approach to automatic counterpoint generation was developed. A team from Michigan used this approach to optimise counterpoints of the 5th species and make them more attractive \cite{polito1997musica}. A similar approach was used in 2004 to generate fugues (hence counterpoint), also using genetic algorithms \cite{garay2004fugue}.  

Many years later, in 2010, a team of researchers from the University of Malaga developed an automated method for the generation of first-order counterpoint using probabilistic logic \cite{Aguilera2010}. Their approach was specifically tailored to compositions in C major, providing a generated counterpoint in response to a given \cf. Please note that this application only evaluates the harmonic attributes of the counterpoint, ignoring the melodic aspect.

Two years later, a team from London developed a way to generate high-quality first-species counterpoint using a variable neighbourhood search algorithm \cite{Herremans2012}. Their research was limited to first-species counterpoint, but they addressed issues such as cost (finding the best counterpoint) and user-friendly interface.

Finally, a research was carried out in 2015 on Fux's counterpoint \cite{komosinski2015automatic}, with the aim of generating the first species counterpoint using dominance relations, has yielded fairly good results. The search demonstrates the use of this paradigm and its applicability, and is a good starting point for composing counterpoints of other species based on the same concept.

\paragraph{}
If we now focus on applications that have gone as far as the user interface and are now ready to be used, we should mention two namesakes, both called 'Counterpointer', which have the merit of offering a functional tool for composing counterpoint.

The first Counterpointer tool \cite{counterpointer_ms}, which anyone\footnote{Anyone... or almost, as it is a paid tool.} can use to check the validity (or otherwise) of their counterpoint. It was last released in 2019 as a desktop application, and it works like this: an apprentice composer tries to write a counterpoint, and then submits it to the tool. The tool then decides whether the counterpoint is valid according to the traditional rules of counterpoint\footnote{Not only Fux's rules, but also those of other authors.}. It also provides feedback to help the student composer improve his or her future counterpoint writing. The tool is not able to write counterpoints automatically, nor is it explicit about how it works, as it is completely closed source and has no accessible report. It is therefore impossible to know the paradigm it uses or the exact rules it follows.

Another attempt at automatic counterpoint writing is the Counterpointer project in 2021, created by a team of students at Brown University as part of a software engineering course \cite{counterpointer_project}. The project is less accomplished than the aforementioned application, but it has the merit of being able to generate two-voice counterpoints of the first, second and third species. It is an entirely free and open source project. While the results are encouraging, the project has been discontinued as it was a course project and their method of finding a counterpoint seems much less efficient than the efficiency that a constraint solver can achieve. 


\paragraph{}
This brief overview leads us to conclude that there is no satisfactory tool for composing counterpoint in a user-friendly way, with good quality, quickly and with several voices. It is to fill this gap that this research has been carried out. This was the aim of T. Wafflard's thesis and it is therefore natural that this thesis should follow in his footsteps.



\section{Standing on the shoulders of giants: underlying works and editions of \gaps used}
As has just been said, this work is the continuation of T. Wafflard's work. However, it also relies heavily on the work of
\begin{itemize}
    \item \textcite{GiLthesis}, who presented an interface for using Gecode functions in Lisp called "GiL". This interface was then tested with some rhythm-oriented constraints.
    \item \textcite{Melothesis}, who explored the use of constraint programming in OpenMusic using GiL. The tool that was produced in this thesis is capable of producing songs with basic harmonic and melodic constraints.
    \item \textcite{Melo2thesis}, who created a tool capable of combining the strengths of the first two implementations while continuing to develop support for GiL.
\end{itemize}

As with T. Wafflard, the musical reference work chosen is Fux's \gap, because it is a pillar of counterpoint theory and because it is fairly easy to extract rules from it (although Fux is sometimes very vague about his intentions). And as with any book published several centuries ago (1725 in the case of \gap), there are many versions and translations. This is good news, as Fux can sometimes be really unclear about what he means, and having many versions (some annotated, some not) from many people who also had to interpret Fux to translate it is a great treasure, as it helps to clarify Fux's meanings. This work is therefore based on several different editions and translations of the book, although it is mainly based on Alfred Mann's English translation \cite{GaPEng}. French (both Chevalier's \cite{GaPFrChevalier} and Denis's \cite{GaPFrDenis}), German \cite{GaPDe} and Latin \cite{GapLa} translations are used when it is necessary to remove an ambiguity or clarify an unclear rule. These translations have been chosen because French is the \textit{lingua franca} of the team; German is the language of Fux and the environment in which he developed; and Latin is the original version, so we can hope that it is the most faithful to what he wanted to convey.



\section{T. Wafflard's thesis in a nutshell}\label{section:thomas-in-a-nutshell}

In 2023, T. Wafflard proposed a complete formalisation of Fux's two-voice counterpoint \cite{wafflard2023}. This formalisation takes each of the rules given by Fux about two-voice composition and translates them into formal logic. Those formal relations are then translated into constraints and given to a constraint solver. When given an input \cf, the solver applies all the constraints it was given and yields an output counterpoint. The following subsection is not intended to be an exhaustive summary of all of T. Wafflard's excellent work, but rather a brief outline of the idea behind it and the procedure followed. This brief overview is given to the reader because many of the concepts in T. Wafflard's work are at the heart of the three-voice generalisation and will be key to understanding the three-voice formalisation.

\subsection{Variables} \label{Wafflard-variables}
In order to formalise Fux's rules, it was necessary to define variables whose purpose is to represent a compositional reality. Thus T. Wafflard created variables to represent various concepts, such as the pitch of notes (written $N$), the harmonic interval between voices (written $H$), the melodic interval from one note to the next (written $M$), and many others. These variables are then related to each other according to the formalised rules. The constraint solver searches for all possible values of these variables, according to the constraints, and stops when all variables in $N$ (the pitches) are fixed, as this means that a solution has been reached (the notes of the counterpoint are known, and this is the goal of the solver).

Useful constants were also defined and will be reused throughout this work. The most important of these are $m$, which represents the length of the \cfs (and thus the number of bars in the composition), $n$, which represents the number of notes in a given counterpoint, or Cons$_{\text{all, p, imp}}$, which represents the set of consonances (all, perfect and imperfect).

Of course, there are many different solutions for the same \cf, and in order to distinguish between two valid solutions (i.e. all solutions that respect all constraints), some costs have been defined. These costs are intended to convey the preferences expressed by Fux in \gap. In fact, Fux sometimes expresses a preference that is not an absolute rule: "this should be done if possible, but is not necessary". Thus, the solver considers a valid solution with a low cost to be better than a valid solution with a high cost. The costs scale from 0 (when we don't care) to 64$m$ (when something should only happen as a last resort), but most costs scale from 0 to 8. These costs are then added together to form a total cost, which the solver tries to minimise.

\subsection{Array notation}
As we have just seen it is necessary to refer to numerous arrays, that formally represent the musical composition and many of its underlying aspects (pitches, harmony, melody, ...). These arrays always have two indexes: the first index represents the time in question, the second index represents the measure in question. These indices are written in computer notation. For example, $X[3, 7]$ represents the value X on the 3rd beat of the 7th measure.

Let's put it all together. Here is a simple example of how the variables and constants are brought together through the formalisation: 
\begin{equation}    
    H[0, m] \in Cons_{\text{perf}}
\end{equation}
 This is a rule which means that the last harmonic interval must be a perfect consonance\footnote{Please refer to section \ref{subsection:modified_variables} for more exact details}. All rules from T. Wafflard's thesis can be found in appendix \ref{appendix:complete-set-of-rule}.



\subsection{In practice}
In practice, to solve this constraint programming problem, the constraints are written in Lisp, and thanks to the Gecode Interface Lisp (GiL) \cite{GiL}, it uses the Gecode constraint solver \cite{Gecode} to find a solution. To make the constraints work, we need a starting point. The starting point is the \cf (see \ref{section:species-counterpoint}).
When given a \cf, the solver defines a set of variables (those mentioned in the formalisation) to which constraints are then applied (the relations from the formalisation) and produces a counterpoint that obeys all the rules that have been defined and whose quality can be given by the cost. As explained in \ref{Wafflard-variables}, in the case of T. Wafflard's implementation, the total cost is the sum of all the costs, and this cost is minimised by a depth first search algorithm that finds the lowest cost, and then gives the corresponding solution.


As for the front-end, all the user sees and interacts with is OpenMusic \cite{OpenMusic}, an object-oriented visual programming environment for musical composition based on Common Lisp \cite{commonlisp} developed by the Parisian institute IRCAM (Institute for Acoustic/Music Research and Coordination) \cite{IRCAM}.

The present work has exactly the same basic functioning. 


\section{The contributions of this paper}
The aim of this work is to generalise T. Wafflard's formalisation to three-voice counterpoint, still based on Fux's work, and to create the corresponding implementation. It would be too easy to believe (wrongly) that three-voice counterpoint is nothing more than the combination of two two-voice counterpoints. From this point of view, we would then calculate a first counterpoint according to the \cf, and then a second counterpoint again according to the \cf, and that's it. Obviously, this view is too simplistic and doesn't really capture all the interactions between three voices. It is to this point (the peculiarities brought about by the addition of a third voice) that a whole section of this thesis will be devoted. Another part will be devoted to discussing and analysing the impact of costs. The following is a more detailed summary of the contributions of this thesis. 
\begin{itemize}
    \item \textbf{Introducing new concepts and redefining variables}:
    As we have just mentioned, a three-part composition is much more than a (two+one) part composition. So we had to redefine and define a whole series of concepts to adapt to this reality. The creation of the (lowest, middle and highest) stratum concept is part of this, and is essential for formalising Fux's counterpoint constraints.
    \item \textbf{Mathematical formalisation of three-part counterpoint}: As with the two parts of the formalisation, we rewrote Fux's explanations into unambiguous English and then translated them into logical notation. This formalisation builds on the previous formalisation for two voices, and sometimes (rarely) has to modify it.
    \item \textbf{Implementation of a working constraint solver for a three-voice composition}: Those logical rules were then implemented as constraints and the solver was adapted to allow a search for two counterpoints.
    \item \textbf{Searching for the best cost management techniques}: Three-part composition introduces so many possibilities for the search that it was important to rethink the way we thought about costs. Therefore, some cost management techniques were discussed to find out the best way to use the costs.
    \item \textbf{Musical analysis of the solutions generated by the solver}: Finding the best solution also means being able to assess the quality of current solutions.
    \item \textbf{Adapted user interface to allow them to compose with three voices and set a cost order}: All the new capabilities of the solver and the costing techniques must also be accessible to the user: it is now possible for a user to freely combine any number of species to form a three-part composition, and to set a cost order to indicate their preferences to the solver (in addition to the previous ability to set personalised costs).
\end{itemize}
% citer Fux page 71 