\chapter{Introduction and context of this work}
\pagenumbering{arabic}
This thesis is a formalisation for three-part counterpoint based on the theory of Johann Joseph Fux, as given in his classic treatise of 1725. It provides a mathematical formalisation of Fux's rules and a computer environment capable of implementing these logical rules in a concrete way to produce Fux-style counterpoint.


This thesis will therefore be divided into several parts: we will first immerse ourselves in \gap, Fux's central work, from which we will meticulously extract the rules laid down by its author. We will briefly discuss these rules to make them unambiguous, and then translate them into formal logic, so that each rule Fux had in mind when writing his work is mathematically recorded. On this basis, we will create a computer implementation using constraint programming. We will then look at how this implementation finds results, discussing the search algorithm and heuristics used. We then discuss the cost techniques used to obtain the best possible results. Finally, we will analyse the musical compositions produced by the tool created.

It is very important to know that this thesis is based on T. Wafflard's thesis "FuxCP: a constraint programming based tool formalizing Fux's musical theory of counterpoint"~\cite{wafflard2023} and article "A Constraint Formalization of Fux's Counterpoint"~\cite{sprockeels2023constraint}. The present work takes up the concepts and definitions of T. Wafflard and could only be understood in its full depth by reading and fully understanding his works as well. This thesis also assumes a basic knowledge of music theory, which can be found in chapter 1 of T. Wafflard's thesis. %A short summary is given in section~\ref{section:thomas-in-a-nutshell}.



\section{A brief history of counterpoint: from Bach to algorithmic generation}
Before delving into the formalities of our study, let's first examine Fux's theory of counterpoint, which forms the basis of the formalisation undertaken in this work. Counterpoint is a compositional technique in which there are several musical lines (or voices) that are independent and distinct from each other, but that are balanced and sound beautiful~\cite{CpSachs}. No voice is dominant over the others, and all are main voices, although some may take a small precedence during part of the composition~\cite{hess2016}.


Counterpoint has been central to the work of many famous composers from different artistic movements, such as Bach in the Baroque era, Mozart in the Classical era and Beethoven in the Romantic era~\cite{kramer1987gradus}. It is still presenet in some modern music~\cite{altozano2017contrapunto}, and has aroused interest over the centuries with the development of key texts on the subject, such as Schenker's Counterpoint~\cite{schenker1906} or Jeppesen's Analysis~\cite{jeppesen1960}. And while Bach mastered counterpoint to an unprecedented level for his time~\cite{yearsley2002}, the central and foundational work in the teaching of counterpoint belongs to another great Baroque composer: the Austrian Johann Joseph Fux and his treatise \gap. In it, this composer gives a detailed analysis of the writing of two-, three- and four-part counterpoint, all narrated as a conversation between a master and his pupil. \gaps is one of the works that deal with species counterpoint\footnote{There are many other types of counterpoint, such as free counterpoint, dissonant counterpoint, linear counterpoint, ...}, a way of conceiving counterpoint in five different types that could then be combined. It is on this work that this dissertation is based.


For Fux, as for many other authors, species counterpoint is governed by many different rules, and it is these rules that interest us in the present work. The rules are based on old concepts that go back to older styles and have been discussed by many other authors~\cite{crocker1962}. Those concepts include, for example, the notions of opposite motion and consonance (which in turn can be either perfect or imperfect). These concepts and their application to counterpoint are particularly interesting because they allow us to consider the composition of counterpoint both in a 'vertical' way, in which we consider the harmony of the notes played together, and in a 'horizontal' way, in which we consider the melodic development of each of the parts individually, which provides the independence of the counterpoints from each other and their melodic beauty.

This is what makes it interesting to analyse from a constraint programming point of view. We'll come back to this later, but for now let's concentrate on Fux's music theory.

\subsection{Fux's theory of counterpoint for two-, three- and four-part composition}
As we have just said, Fux uses a set of rules to create 'the correct counterpoint'. These rules can be divided into three categories: melodic rules, harmonic rules and motion rules. We will examine them here, bearing in mind that the formalisation of all the rules for two-voice counterpoint can be found in T. Wafflard's thesis, and that the complete set of rules (in mathematical form) can be found in Appendix~\ref{appendix:complete-set-of-rule} of this thesis.

\subsubsection{Melodic Rules}

Fux explains that there are rules that apply within parts (the horizontal rules) about the order of the interval between one note and the next: we find, for example, that a melody is more "beautiful"\footnote{Throughout this work we will speak of the "beauty of music". This beauty is highly subjective, and therefore reference will be made to the Fuxian concept of music to define whether a melody is beautiful or not. In other words, music is considered beautiful if it conforms to Fux's rules, and vice versa.} when the intervals between its successive notes are small, when there is no chromatic succession, when the successive notes are varied, and so on. These 'horizontal' rules are called 'melodic' rules because they concern only the melody. These rules apply within a given voice.


\subsubsection{Harmonic Rules}


If there is a horizontal perspective to counterpoint, there is also, of course, a vertical perspective. This perspective is expressed in a harmonic relationship between the different voices. At each point in the composition, a series of rules apply that concern harmony alone (known as harmonic rules). Here are some harmonic rules from Fux, given as an example: the interval between the voices must be a consonance\footnote{A third, a fifth, a sixth or an octave.} on the first beat of each measure; imperfect consonances\footnote{Thirds and sixths.} are preferred to fifths, which in turn are preferred to octaves; and the voices can't use the same note at the same time. These rules apply between the voices.

\subsubsection{Motion Rules}
Finally, there is a third type of rule: motion rules. These rules are a hybrid of the two discussed above in that they consider not only vertical interaction, i.e. harmony, but also horizontal interaction, i.e. melody. They can therefore be seen as 'diagonal' rules that relate the unique melody of each counterpoint to its respective harmonies. Here are some motion rules, given as an example: voices that move in opposite directions are preferred (i.e. if one voice goes up, we want the other to go down), there should be no successive fifths or successive octaves between voices, and a direct motion should not lead to a perfect consonance. As you can see, these rules take into account not just two voices at a given point, but over several measures. They ensure that the voices move well together.

\subsubsection{Preferences}\label{subsection:preferences-vs-hard-rules}
This last point is one of the most important in Fux's theory, namely that of preferences. Preferences are hints that Fux gives in his work in order to write even better counterpoints. As their name suggests, preferences are optional and not compulsory to follow, as other strict rules would be. However, preferences are crucial to Fux's work because they allow us to distinguish between two valid solutions (those that obey all the strict rules) and decide which is the best, thus allowing the composer to control how the theory is applied.

Fux is never clear about whether a rule\footnote{We use the generic term "rule" to refer to both mandatory rules and preferences.} is a preference or a strict rule --- and that's normal, what he conveys is mostly intuition, and human beings are quite capable of understanding whether a rule is a preference or an obligation; Fux probably didn't expect someone to try to formalise his work three centuries later.

These preferences should be respected whenever possible, and if not, so be it. Here's a good example: Fux indicates that we prefer to have as many different notes as possible in the composition. This is not an absolute rule, but a preference. The more variety there is in the composition, the more beautiful it will be, and the more preferable it will be.

\subsection{Species counterpoint}\label{section:species-counterpoint}
When we talk about species counterpoint, we're talking about five categories of counterpoint. Each species is a separate concept, each with its own peculiarities. We'll go into more detail about the different species below. Let's focus first on how species counterpoint works. When composing counterpoint, the starting point is a fixed melodic line, the \cf, which is a basic melody composed entirely of whole notes. It is the basis of a composition when writing counterpoint. It is from this voice and in relation to it that the others are composed. It is important to note that once the composition is complete, the \cfs is neither more nor less important than the other voices, and it has the same melodic independence as the other voices. It is, therefore, nothing more than the basis from which we begin to write.

Let's take a look at the five species:
\begin{enumerate}
    \item \textbf{First species}: Note against note \textendash{} the counterpoint is composed entirely of whole notes, and the composition is a sequence of harmonies sounding on the first beat between the counterpoint and the other voices.
    \item \textbf{Second species}: Half notes against whole notes \textendash{} the counterpoint is composed entirely of half notes, which introduce dissonant harmonies.
    \item \textbf{Third species}: Quarters against whole notes \textendash{} the counterpoint is made up entirely of quarter notes, which allow more different movements and more freedom in the composition.
    \item \textbf{Fourth species}: The ligature \textendash{} the counterpoint is delayed by two beats, creating syncopation. The notes are all round (i.e. tied half notes, since they span between two measures).
    \item \textbf{Fifth species}: Florid counterpoint \textendash{} the counterpoint is a mixture of all the other species and is the richest form of counterpoint. It allows great freedom of composition while respecting the rules of the other types.
\end{enumerate}

Although these types could technically be combined to form a three-part composition with two different types of counterpoint, Fux seems to prefer to write a 'special' counterpoint (i.e. second, third, fourth or fifth species) in one voice and only whole notes (i.e. a first species counterpoint) in another voice. He also sometimes says that it is possible and recommended to mix species, but does not do so extensively.

\subsection{Software tool for writing species counterpoint}
We have been discussing the longstanding tradition of counterpoint, a musical technique shaped by countless generations of composers. As technology advanced, the idea of automating counterpoint composition emerged. An early attempt, by Schottstaedt in 1984~\cite{bill1984}, involved an expert system based on Fux's rules. His approach used over 300 if-else clauses, but this method had obvious limitations compared to what modern constraints are capable of.

Indeed, it's crucial to understand that if-else clauses are unidirectional, whereas constraints are bidirectional. Schottstaedt's algorithm relied heavily on specific conditions, captured by numerous if-else clauses. In contrast, modern constraint systems support bidirectional constraints with multiple arguments. These systems don't just find single solutions, they represent sets of potential solutions. This flexibility is a significant improvement over the directional nature of if-else clauses.

Furthermore, constraint systems offer an advantage in specifying intricate search heuristics. This adaptability and efficiency highlight the stark contrast between the outdated approach of if-else clauses and the modern capabilities of bidirectional constraint systems in the realm of counterpoint composition.

\paragraph{}
In 1997, a genetic programming and symbiosis approach to automatic counterpoint generation was developed by J. Polito et al. This team from Michigan used the genetic approach to optimise counterpoints of the 5th species and make them more attractive~\cite{polito1997musica}. A similar approach was used in 2004 to generate fugues (hence counterpoint), also using genetic algorithms~\cite{garay2004fugue}. The results are quite promising, and generate more than interesting results, but the end result is still far from being able to provide a complete counterpoint composition.

Many years later, in 2010, G. Aguilera et al., from the University of Malaga, developed an automated method for the generation of first-species counterpoint using probabilistic logic~\cite{Aguilera2010}. Their approach was specifically tailored to compositions in C major, providing a generated counterpoint in response to a given \cf. Please note that this application only evaluates the harmonic attributes of the counterpoint, ignoring the melodic aspect.

Two years later, D. Herremans and K. Sörensen developed a way to generate high-quality first-species counterpoint using a variable neighbourhood search algorithm~\cite{Herremans2012}. Their research was limited to first-species counterpoint, but they addressed issues such as preferences (finding the best counterpoint) and user-friendly interface. Once again, their results are more than impressive, but their research is limited to the first two-voice species.

Finally, a research was carried out in 2015 on Fux's counterpoint~\cite{komosinski2015automatic}, with the aim of generating the first species counterpoint using dominance relations, has yielded fairly good results. The search demonstrates the use of this paradigm and its applicability, and is a good starting point for composing counterpoints of other species based on the same concept.

\paragraph{}
If we now focus on applications that have gone as far as the user interface and are now ready to use, we should mention two namesakes, both called 'Counterpointer', which have the merit of offering a functional tool for composing counterpoint.

The first Counterpointer tool~\cite{counterpointer_ms}, which anyone\footnote{Anyone... or almost, as it is a paid tool.} can use to check the validity (or not) of their counterpoint. Its last release was in 2019 as a desktop application, and it works like this: an apprentice composer tries to write a counterpoint, and then submits it to the tool. The tool then decides whether the counterpoint is valid according to the traditional rules of counterpoint\footnote{Not only Fux's rules, but also those of other authors.}. It also provides feedback to help the student composer improve their future counterpoint writing. The tool is not able to write counterpoints automatically, nor is it explicit about how it works, as it is completely closed source and has no accessible report. It is therefore impossible to know the paradigm it uses or the exact rules it follows.

Another attempt at automatic counterpoint writing is the Counterpointer project in 2021, created by a team of students at Brown University as part of a software engineering course~\cite{counterpointer_project}. The project is less accomplished than the aforementioned application, but it has the merit of being able to generate two-voice counterpoints of the first, second and third species. It is an entirely free and open source project. While the results are encouraging, the project has been discontinued as it was a course project and their method of finding a counterpoint seems much less efficient than the efficiency that a constraint solver can achieve. 


\paragraph{}
This brief overview leads us to conclude that there is no satisfactory tool for composing counterpoint in a user-friendly way, with good quality, quickly and with several voices. It is to fill this gap that this research has been carried out. This was the aim of T. Wafflard's thesis and it is therefore natural that this thesis should follow in his footsteps.



\section{Standing on the shoulders of giants: underlying works and editions of \gaps used}
As has been said, this work is the continuation of T. Wafflard's work. However, it also relies heavily on the work of:
\begin{itemize}
    \item \textcite{GiLthesis}, who presented an interface for using Gecode functions in Lisp called "GiL". This interface was then tested with some rhythm-oriented constraints.
    \item \textcite{Melothesis}, who explored the use of constraint programming in OpenMusic using GiL. The tool that was produced in this thesis is capable of producing songs with basic harmonic and melodic constraints.
    \item \textcite{Melo2thesis}, who created a tool capable of combining the strengths of the first two implementations while continuing to develop support for GiL.
\end{itemize}

As with T. Wafflard, the musical reference work chosen is Fux's \gap, because it is a pillar of counterpoint theory and because it is fairly easy to extract rules from it, although Fux is sometimes very vague about his intentions. And as with any book published several centuries ago (1725 in the case of \gap), there are many versions and translations. This is good news, as Fux can sometimes be really unclear about what he means, and having many versions (some annotated, some not) from many people who also had to interpret Fux to translate it is a great treasure, as it helps to clarify Fux's meanings. This work is therefore based on several different editions and translations of the book, although it is mainly based on Alfred Mann's English translation~\cite{GaPEng}. French (both Chevalier's~\cite{GaPFrChevalier} and Denis's~\cite{GaPFrDenis}), German~\cite{GaPDe} and Latin~\cite{GapLa} translations are used when it is necessary to remove an ambiguity or clarify an unclear rule. These translations have been chosen because French is the lingua franca of the team; German is the language of Fux and the environment in which he evolved; and Latin is the original version, so we can hope that it is the most faithful to what he wanted to convey.

\section{Tools and implementation}\label{subsection:tools-and-implementation}
This subsection discusses the implementation of the automatic counterpoint generator. To do so, we briefly explain how constraint programming works, the tools used by FuxCP, and the actual implementation of it.

\subsection{Constraint Programming}
Constraint Programming (CP) is a programming paradigm used to solve large combinatorial problems, such as planning and scheduling problems. It works by defining constraints between variables that limit the values these variables can potentially take~\cite{rossi2008constraint}. In doing so, the domains of these variables are reduced. R. Barták explains in a very clear way what a constraint really is, as he describes in his "Guide to Constraint Programming"~\cite{bartak1998constraint}:

\begin{quote}
    A constraint is simply a logical relation among several unknowns (or variables), each taking a value in a given domain. A constraint thus restricts the possible values that variables can take, it represents some partial information about the variables of interest. For instance, "the circle is inside the square" relates two objects without precisely specifying their positions, i.e., their coordinates. Now, one may move the square or the circle and he or she is still able to maintain the relation between these two objects. Also, one may want to add other object, say triangle, and introduce another constraint, say "square is to the left of the triangle". From the user (human) point of view, everything remains absolutely transparent.
\end{quote}

The question now is: what is the connection between this problem-solving method and counterpoint composition? Well, it turns out that music is actually a very good application of constraint programming: you can represent all aspects of a composition by variables, set rules between these variables, and the solver takes care of finding a valid solution. In fact, in music, it's never just one factor that determines whether the music is beautiful, but an interaction of many. And as humans, it's sometimes difficult to find a valid solution (i.e. to compose music that sounds good) because the range of possibilities and the interactions between factors are so numerous.  However exploring a search space in which a large number of constraints are defined is something that a constraint solver does very well. The most arduous task then becomes putting the rules that make music beautiful down on paper, and this is the task that many musicologists and composers, like Fux, have set themselves. Once these rules have been defined, it is "simply" a matter of formalising them and passing them to the constraint solver so that it can compose a melody that respects these rules.

What's more, by defining a rigorous way of distinguishing a good composition from a bad one, the solver can even find increasingly beautiful solutions.
\paragraph{}
To make sure that Constraint Programming is a well understood concept, we here review its main concepts:

\paragraph{Constraint propagation}
Each time a constraint is defined, the domain of the affected variables is reduced according to the possibilities left by the constraint. This is called constraint propagation. Let's imagine a variable $A$ whose domain is $\{0, 1, 2, 3\}$ and a variable $B$ whose domain is $\{0, 1, 2\}$. When the constraint $A<B$ is applied, the domain of $A$ is reduced to $\{0, 1\}$, because the new constraint makes it impossible for $A$ to have a value of $2$ or $3$ without violating the constraint.

A constraint solver is an implementation that systematically searches the search space for a solution. A given problem may have many solutions, only one, or none at all.

A solution is found when all variables are fixed, i.e. their domain is reduced to a single value. We know that a problem has no solution when the domain of any variable becomes empty (because this means that no value can be found for that variable without violating a constraint).

\paragraph{Branching}
Obviously, declaring constraints is not enough to magically find a solution. In the example we proposed earlier (with A and B), the single constraint placed on the search space doesn't allow us to determine the values of A or B, and their domains remain composed of more than one value. 
To actually find a solution, the constraint performs a branching. That is, it studies two antagonic possibilities and splits the search space into two subproblems accordingly. More specifically, it chooses a variable and studies the case where that variable is equal to a certain value, and the case where it is not equal to that value. For example, the solver might study the case where $B=0$ (the first branch) and the case where $B \neq 0$ (the second branch). If the solver finds an inconsistency in either case, it knows that the entire branch can be discarded (as it does not lead to a valid solution). Immediately after branching, the solver again performs constraint propagation, since constraint propagation occurs whenever any domain is modified, and consists of adjusting all domains to which the modified domain is linked by a constraint. In our case, after setting $B$ to $0$, the solver propagates all constraints linked to $B$, i.e. the only constraint in our problem (i.e. $A<B$). This affects the domain of $A$, reducing it to an empty domain (because no value in the domain of $A$ is less than $0$). The solver, noticing that one domain is empty, concludes that this branch contains no solution and therefore knows that the only possible branch is the other one, i.e. the one in which we assumed $B\neq 0$. We can therefore safely remove $0$ from the domain of $B$, since this value is not contained in any solution. Repeating the branching process each time it is necessary produces three solutions: $A=0$ and $B=1$, $A=0$ and $B=2$, and finally $A=1$ and $B=2$.

\paragraph{Heuristics}
As with all problems involving searching a space in quest of a solution, it is very useful to have heuristics allowing for an efficient search. A heuristic is a rule or strategy used to make informed decisions about variable assignments and value choices during the solution search. These rules are designed to exploit the characteristics and structures of the problem to improve the chances of finding solutions more quickly. A common heuristic is variable ordering, where the algorithm selects variables to assign values to based on factors such as the size of the domain or the number of associated constraints. Another important heuristic is value ordering, which determines the order in which values are tested for a given variable assignment. By incorporating heuristics, constraint solvers can prioritise the most promising branches of the search tree, effectively reducing the search space and speeding up the identification of feasible solutions. While heuristics speed up the solving process, it's important to strike a balance between exploration and exploitation, as overly aggressive heuristics risk missing potentially valuable solution paths.


To return to our previous example, a value-ordering heuristic might be "branch first on low values of $A$ and high values of $B$, since we know that we are looking for a solution where $A$ is less than $B$, and we can reason that there are more chances of satisfying this constraint when $B$ is large and $A$ is small.

\paragraph{Optimal solutions}
Constraint programming can also be used to find optimal solutions, i.e. solutions that have minimum cost with respect to a cost function that assigns a cost to each solution.

For example, in our simple example, we could define cost as the sum of $A$ and $B$ and want to minimise the cost. This means that we are looking for a valid solution where $A$ and $B$ are as small as possible. In this case, only the first of the three solutions mentioned above is chosen, i.e. $A=0$ and $B=1$, as it is the best possible solution (with a cost of 1).

\paragraph{Branch and Bound}
Branch and Bound is a systematic algorithm used in Constraint Programming to efficiently explore the solution space and find an optimal solution with respect to a given cost or objective function. This technique extends the basic Constraint Programming approach by introducing a mechanism to prune unpromising branches of the search tree, thereby reducing the computational effort required to find the optimal solution. The algorithm starts with the initial problem and iteratively divides it into subproblems, called branches, by making decisions about variable assignments. For each branch, the algorithm evaluates its feasibility and its potential to lead to a better solution. If a branch is deemed infeasible or cannot possibly improve on the current best-known solution, it is pruned from further consideration. This process continues until all branches have been explored, or until the algorithm converges on the optimal solution. Branch and Bound is particularly valuable for large combinatorial problems because it efficiently narrows the search space, allowing the solver to focus on promising regions and accelerating the discovery of optimal solutions in constrained programming scenarios~\cite{morrison2016branch}. 


\subsection{OpenMusic}
OpenMusic is a powerful and innovative visual programming environment, written in CommonLisp, designed specifically for composers, researchers and musicians involved in computer-aided composition~\cite{OpenMusic}. Developed by the Institute for Research and Coordination in Acoustics/Music (IRCAM) in Paris, OpenMusic provides a graphical interface that allows users to create and manipulate musical structures using a variety of predefined modules. This visual programming environment facilitates the representation of complex musical ideas, algorithms and data flows through a user-friendly interface, making it accessible to both novice and experienced composers. 


FuxCP is a library for OpenMusic, which means that OpenMusic is the interface for using FuxCP. Any user wishing to use FuxCP writes their \cfs (FuxCP's input) into OpenMusic and then launches the solution search from within OpenMusic. Specifically, FuxCP retrieves the \cfs from OpenMusic and then defines the constraint problem. When a solution (the counterpoints) is found, it is passed to OpenMusic and the user gets it as an OpenMusic object.

\subsection{GiL and Gecode}
GiL is an interface between OpenMusic and Gecode, which in turn is an open source toolkit for developing constraint-based systems~\cite{Gecode}, which provides a high-level C++ library for efficiently modelling and solving constraint problems. GiL allows to use the Gecode tool within a Lisp environment, thanks to the Common Foreign Function Interface (CFFI)~\cite{CFFI}.

\subsection{Concrete implementation}
To put it all together: FuxCP gets its input (the \cf) from the user interface, which is OpenMusic. It then defines a constraint programming problem in Gecode, using GiL. 

As for the way it defines the problem, here is a little clarification: first, when the \cfs is received, a whole series of constants and variables are defined: for example, the length of the \cf, the arrays representing the pitches of the counterpoints, ...
Then all these variables are constrained according to the constraints defined in the formalisation of Fux's rules (discussed in chapter~\ref{chapter:species}). The constraints are set sequentially: first the constraints on the \cf, then the constraints on the first counterpoint, and finally the constraints on the second counterpoint. A diagram of the code architecture and the integration of FuxCP with other tools can be found in Appendix~\ref{fig:softwarearchitecure}.

%\section{T. Wafflard's thesis in a nutshell}\label{section:thomas-in-a-nutshell}

%In 2023, T. Wafflard proposed a complete formalisation of Fux's two-voice counterpoint~\cite{wafflard2023}. This formalisation takes each of the rules given by Fux concerning two-voice composition and translates them into formal logic. Those formal relations are then translated into constraints and given to a constraint solver. When given a \cfs as an input, the solver applies all the constraints it was given and returns a counterpoint as an output. The following subsection is not intended to be an exhaustive summary of all of T. Wafflard's excellent work, but rather a brief outline of the idea behind it and the procedure followed. This brief overview is given to the reader because many of the concepts in T. Wafflard's work are at the heart of the three-voice generalisation and will be key to understanding the three-voice formalisation.



%\subsection{In practice}
%In practice, to solve this constraint programming problem, the constraints are written in Lisp. Thanks to the Gecode Interface Lisp (GiL)~\cite{GiL}, the constraints are passed to the Gecode constraint solver~\cite{Gecode} to find a solution. To make the constraints work, we need a starting point. The starting point is the \cf (see~\ref{section:species-counterpoint}).
%When given a \cf, the solver defines a set of variables (those mentioned in the formalisation) to which constraints are then applied (the relations from the formalisation) and produces a counterpoint that obeys all the rules that have been defined and whose quality can be given by the cost. As explained in~\ref{Wafflard-variables}, in the case of T. Wafflard's implementation, the total cost is the sum of all the costs, and this cost is minimised by a depth first search algorithm that finds the lowest cost, and then gives the corresponding solution.
 
 
%As for the front-end, all the user sees and interacts with is OpenMusic~\cite{OpenMusic}, an object-oriented visual programming environment for musical composition based on Common Lisp~\cite{commonlisp} developed by the Parisian institute IRCAM (Institute for Acoustic/Music Research and Coordination)~\cite{IRCAM}.
 
%Everything that has been said about T. Wafflard's thesis also applies to this thesis. The present thesis, as well as the implementation it provides, is an extension of T. Wafflard's thesis and is fully compatible with it.

\section{The contributions of this thesis}
The aim of this work is to generalise T. Wafflard's formalisation to three-voice counterpoint, still based on Fux's work, and to create the corresponding implementation. It would be too easy to believe (wrongly) that three-voice counterpoint is nothing more than the combination of two two-voice counterpoints. Following this point of view, we would then calculate a first counterpoint according to the \cf, and then a second counterpoint again according to the \cf, and that's it. Unfortunately, this view is too simplistic and doesn't really capture all the interactions between three voices.
An entire chapter of this thesis is devoted to the unique characteristics that arise from the introduction of a third voice.
Another chapter is dedicated to translating the rules from \gaps into formal logic. A final but not less important section discusses and analysing the impact of costs, and the musicality of the solutions. The following is a more detailed summary of the contributions of this thesis. 
\begin{itemize}
    \item \textbf{Concepts and variables to three-part counterpoint}:
    As we have just mentioned, a three-part composition is much more than a (two+one) part composition. So we define a whole series of concepts to adapt to this reality. The creation of the (lowest, middle and highest) stratum concept is part of this, and is essential for formalising Fux's counterpoint constraints. All of this is discussed in Chapter~\ref{chapter:defining-some-concepts-and-redifing-the-variables}.
    \item \textbf{Mathematical formalisation of three-part counterpoint}: As with the two parts of the formalisation, we rewrote Fux's explanations into unambiguous English and then translated them into logical notation. This formalisation builds on the previous formalisation for two voices, and sometimes (rarely) has to modify it. This formalisation can be found in Chapter~\ref{chapter:species}.
    \item \textbf{Implementation of a working constraint solver for a three-voice composition}: Those logical rules were then implemented as constraints and the solver was adapted to allow a search for two counterpoints. The whole code of this implementation can be found in Appendix~\ref{chapter:whole-code}, and its architecture in the Appendix~\ref{chapter:architecture}.
    \item \textbf{Researching the best way to express Fux's preferences}: Three-part composition introduces so many possibilities for result composition that it is important to rethink the way we think about preferences. These preferences are understood by the solvers as costs (where a preferred solution in Fux's sense has a lower cost to the solver). Therefore, some techniques for managing these preferences are discussed to find out the best way to implement them as costs. This is very important as it allows the solver to produce solutions with high musicality. These techniques are discussed in Chapter~\ref{chapter:search}.
    \item \textbf{Musical analysis of the solutions generated by the solver}: Finding the best solution also means being able to assess the quality of current solutions. For further details, please refer to Chapter~\ref{chapter:musicality}. 
    \item \textbf{User interface for three-point counterpoint that allows a composer to specify how preferences are used in the solver.}: All the new capabilities of the solver and the costing techniques must also be accessible to the user: it is now possible for a user to freely combine any number of species to form a three-part composition, and to set a cost importance order to indicate their preferences to the solver (in addition to the already existing ability to set personalised costs). A guide to its installation and use can be found in Appendix~\ref{chapter:user-guide}.
\end{itemize}


% citer Fux page 71 