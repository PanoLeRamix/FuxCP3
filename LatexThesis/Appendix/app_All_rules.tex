\chapter{Complete set of rules for two and three part compositions}\label{appendix:complete-set-of-rule}
\section*{Constraints of the First Species}
\subsection*{Harmonic Constraints of the First Species}
\begin{enumerate}[wide, label=\bfseries 1.H\arabic*]
  \item\label{rule:allcons}{\textit{All harmonic intervals must be consonances.}} 
\begin{equation}
    \begin{gathered}
        \forall j \in [0, m)\quad 
        H[0, j] \in Cons
    \end{gathered}
\end{equation}
This can be expressed with the constraint {\small\texttt{(gil::g-member *sp* ALL\_CONS\_VAR h-intervals)}} (see original code for more details).

\item\label{rule:firstpcons}{\textit{The first harmonic interval must be a perfect consonance.}}
When dealing with two-part composition:
\begin{equation}
    \begin{gathered}
        H[0, 0] \in Cons_{p}
    \end{gathered}
\end{equation}

\item\label{rule:lastpcons}{\textit{The last harmonic intervals must be a perfect consonance.}}
When dealing with three-part composition:
\begin{equation}
  \begin{gathered}
      H[0, m-1] \in Cons_{p}
  \end{gathered}
\end{equation}

\item\label{rule:keytone}{\textit{The key tone is tuned according to the first note of the \cfdot}}

\begin{equation}
    \begin{gathered}
        \lnot IsCfB[0, 0] \implies H[0, 0] = 0\\
        \lnot IsCfB[0, m-1] \implies H[0, m-1] = 0
    \end{gathered}
\end{equation}

\item{\textit{The counterpoint and the \cf cannot play the same note at the same time except in the first and last measure.}}

\begin{equation}
    \begin{gathered}
        \forall j \in [1, m-1)\quad
        Cp[0, j] \neq Cf[j]
    \end{gathered}
\end{equation}

\item{\textit{Imperfect consonances are preferred to perfect consonances.}}


\begin{equation}
    \begin{gathered}
        \forall j \in [0, m)\\
        Pcons_{costs}[j] = \begin{cases}
            cost_{Pcons} & \text{if } H[0, j] \in Cons_{p}\\
            0 & \text{otherwise}
        \end{cases}\\
        \text{moreover } \C = \C \cup \sum _{c \in Pcons_{costs}} c
    \end{gathered}
\end{equation}

\item{and \textbf{1.H8} \textit{The harmonic interval of the penultimate note must be a major sixth or a minor third depending on the \cf pitch.}}
\addtocounter{enumi}{1} 
\begin{equation}
    \begin{gathered}
        % \lnot IsCfB[0, m-2] \implies H[0, m-2] = 9
        \rho := \max (positions(m)) - 1\\
        H[\rho] = \begin{cases}
            9 & \text{if } IsCfB[\rho]\\
            3 & \text{otherwise}
        \end{cases}\\
        \text{where } \rho \text{ represents the penultimate index of any counterpoint.}
    \end{gathered}
\end{equation}

\subsection*{Melodic Constraints of the First Species}
\end{enumerate}

\begin{enumerate}[wide, label=\bfseries 1.M\arabic*]
\item\label{rule:notritone}{\textit{Tritone melodic intervals are forbidden.} }

\begin{equation}
    \begin{gathered}
        \forpm\\
        M[\rho] = 6 \implies Mdeg_{costs}[\rho] = cost_{tritoneMdeg}\\
    \end{gathered}
\end{equation}

\item\label{rule:mlesixth}{\textit{Melodic intervals cannot exceed a minor sixth interval.}}

\begin{equation}
    \begin{gathered}
        \forj\quad
        M[0, j] \leq 8
    \end{gathered}
\end{equation}

\subsection*{Motion Constraints of the First Species}
\end{enumerate}
\begin{enumerate}[wide, label=\bfseries 1.P\arabic*]

\item\label{rule:nopconsbydm}{ \textit{Perfect consonances cannot be reached by direct motion.}}

When dealing with two-part composition:
\begin{equation}
    \begin{gathered}
        \forj\quad
        H[0, j+1] \in Cons_{p} \implies P[0, j] \neq 2
    \end{gathered}
\end{equation}

When dealing with three-part composition:
\begin{equation} \begin{aligned}
  &\forall j \in [0, m-2) :\\
  &P[0, j] = 2 \land H[0, j+1] \in Cons_{p} \\
  &\iff cost_{\text{{direct\_move\_to\_p\_cons}}}[j] = 8
\end{aligned} \end{equation}

\item\label{rule:codmotions} {\textit{Contrary motions are preferred to oblique motions which are preferred to direct motions.}}

\begin{multicols}{3}
    \begin{itemize}
        \item $cost_{con}$\\ \dft{no cost}
        \item $cost_{obl}$\\ \dft{low cost}
        \item $cost_{dir}$\\ \dft{medium cost}
    \end{itemize}
\end{multicols}

\begin{equation}
    \begin{gathered}
        \forj\\
        P_{costs}[j] = \begin{cases}
            cost_{con} & \text{if } P[0, j] = 0\\
            cost_{obl} & \text{if } P[0, j] = 1\\
            cost_{dir} & \text{if } P[0, j] = 2
        \end{cases}\\
        \text{moreover } \C = \C \cup \sum _{c \in P_{costs}} c
    \end{gathered}
\end{equation}

\item\label{rule:battuta}{ \textit{At the start of any measure, an octave cannot be reached by the lower voice going up and the upper voice going down more than a third skip.}}


\begin{equation}
    \begin{gathered}
        i := \max (\B), \forj\\
        H[0, j+1] = 0 \land P[i, j] = 0 \land \begin{cases}
            M_{brut}[i, j] < -4 \land IsCfB[i, j] \iff \bot\\
            M_{cf}[i, j] < -4 \land \lnot IsCfB[i, j] \iff \bot
        \end{cases}\\
        \text{where } i \text{ stands for the last beat index in a measure.}
    \end{gathered}
    \label{eq:battuta}
\end{equation}
\end{enumerate}

\section*{Constraints of the Second Species}
\subsection*{Harmonic Constraints of the Second Species}
\begin{enumerate}[wide, label=\bfseries 2.H\arabic*]
\item\label{rule:consthesis}{ \textit{Thesis harmonies cannot be dissonant.}}

As explained above, there is no constraint to add because it would be a duplicate of rule \ref{rule:allcons}.

\item\label{rule:arsisdim}{\textit{Arsis harmonies cannot be dissonant except if there is a diminution.}}

\begin{equation}
    \begin{gathered}
        \forj\\
        IsDim[j] = \begin{cases}
            \top & \text{if } M^2[0, j] \in \{3, 4\} \land M^1[0, j] \in \{1, 2\} \land M^1[2, j] \in \{1, 2\}\\
            \bot & \text{otherwise}
        \end{cases}
    \end{gathered}
\end{equation}

\begin{equation}
    \begin{gathered}
        \forj \quad
        \lnot IsCons[2, j] \implies IsDim[j]
    \end{gathered}
\end{equation}

\item\label{rule:penult2nd} \label{rule:penultexception}{and \textbf{2.H4} \textit{In the penultimate measure the harmonic interval of perfect fifth must be used for the thesis note if possible. Otherwise, a sixth interval should be used instead.}}
\addtocounter{enumi}{1}

\begin{equation}
    \begin{gathered}
        H[0, m-2] \in \{7, 8, 9\}\\
        \therefore penulthesis_{cost} = \begin{cases}
            cost_{penulthesis} & \text{if } H[0, m-2] \neq 7\\
            0 & \text{otherwise}
        \end{cases}\\
        \text{moreover } \C = \C \cup penulthesis_{cost}
    \end{gathered}
\end{equation}

\end{enumerate}
\subsection*{Melodic Constraints of the Second Species}
\begin{enumerate}[wide, label=\bfseries 2.M\arabic*]

\item\label{rule:octaveleap}{ \textit{If the two voices are getting so close that there is no contrary motion possible without crossing each other, then the melodic interval of the counterpoint can be an octave leap.}}

\begin{equation}
    \begin{gathered}
        \forj, \forall M_{cf}[j] \neq 0\\
        M[0, j] = 12 \implies (H_{abs}[0, j] \leq 4) \land (IsCfB[j] \iff M_{cf}[j]>0)
    \end{gathered}
\end{equation}

\item\label{rule:notsamecons}{ \textit{Two consecutive notes cannot be the same.}}
When dealing with two-part composition:
\begin{equation}
    \begin{gathered}
        \forp \quad
        Cp[\rho] \neq Cp[\rho+1]
    \end{gathered}
\end{equation}

When dealing with three-part composition:
\begin{equation}
  \begin{aligned}
      &\forall j \in [1, m-1), \quad j \neq m-2:\\
      &((N[2, j-1] \neq N[0, j]) \land (N[0, j] \neq \land N[2, j])) \\
      &\land \\
      & ((N[2, m-3] \neq N[0, m-2]) \lor (N[0, m-2] \neq N[2, m-2]) )
  \end{aligned}
\end{equation}


\end{enumerate}
\subsection*{Motion Constraints of the Second Species}
\begin{enumerate}[wide, label=\bfseries 2.P\arabic*]

\item\label{rule:motion2nd}{\textit{If the melodic interval of the counterpoint between the thesis and the arsis is larger than a third, then the motion is perceived based on the arsis note.}}


\begin{equation}
    \begin{gathered}
        \forj \quad
        P_{real}[j] = \begin{cases}
            P[2, j] & \text{if } M[0, j] > 4\\
            P[0, j] & \text{otherwise}
        \end{cases}
    \end{gathered}
\end{equation}

\item\label{rule:battuta2}{ \textit{Rule \ref{rule:battuta} on the battuta octave is adapted such that it focuses on the motion from the note in arsis.}}
\item 
This constraint already had an adapted mathematical notation in the chapter of
the first species. Note that this constraint would indeed use P[2] and not P$_{real}$.


\end{enumerate}
%%%%%%%%%%%%%

\section*{Constraints of the Third Species}
\subsection*{Harmonic Constraints of the Third Species}
\begin{enumerate}[wide, label=\bfseries 3.H\arabic*]
  \item\label{rule:fivequarters}{ \textit{If five notes follow each other by joint degrees in the same direction, then the harmonic interval of the third note must be consonant.}}

\begin{equation}
    \begin{gathered}
        \forj\\
        % \{M[0, j]\land M[1, j]\land M[2, j]\land M[3, j]\} \leq 2\ \land\\
        % \left(
        %     \{M_{brut}[0, j]\land M_{brut}[1, j]\land M_{brut}[2, j]\land M_{brut}[3, j]\} > 0\ \lor \right. \\
        %     \left.
        %     \{M_{brut}[0, j]\land M_{brut}[1, j]\land M_{brut}[2, j]\land M_{brut}[3, j]\} < 0\
        % \right)\\
        % \implies IsCons[2, j]
        \left(
            \bigwedge_{i=0}^{3} M[i, j] \leq 2
        \right)
        \land
        \left(
            \bigwedge_{i=0}^{3} M_{brut}[i, j] > 0
            \lor
            \bigwedge_{i=0}^{3} M_{brut}[i, j] < 0
        \right)\\
        \implies IsCons[2, j]
    \end{gathered}
\end{equation}


\item\label{rule:thirddiss} {\textit{If the third harmonic interval of a measure is dissonant then the second and the fourth interval must be consonant and the third note must be a diminution.}}


\begin{equation}
    \begin{gathered}
        \forj\\
        IsCons[2, j] \lor \left( IsCons[1, j] \land IsCons[3, j] \land IsDim[j]\right)\\
        \text{where } IsDim[j]=\top \text{ when the \nth{3} note of the measure } j \text{ is a diminution.}
    \end{gathered}
\end{equation}

\item\label{rule:cambiata} {\textit{It is best to avoid the second and third harmonies of a measure to be consonant with a one-degree melodic interval between them.}}


\begin{equation}
    \begin{gathered}
        \forj\\
        Cambiata_{costs}[j] = \begin{cases}
            cost_{Cambiata} & \text{if } IsCons[1, j] \land IsCons[2, j] \land M[1, j] \leq 2\\
            0 & \text{otherwise}
        \end{cases}
    \end{gathered}
\end{equation}

\item\label{rule:penult3sp} {\textit{In the penultimate measure, if the \cf is in the upper part, then the harmonic interval of the first note should be a minor third.}}

\begin{equation}
    \begin{gathered}
        \lnot IsCfB[m-2] \implies H[0, m-2] = 3
    \end{gathered}
\end{equation}
\end{enumerate}
\subsection*{Melodic Constraints of the Third Species}
\begin{enumerate}[wide, label=\bfseries 3.M\arabic*]
  \item\label{rule:twobeats} {\textit{Each note and its two beats further peer are preferred to be different.}}


\begin{equation}
    \begin{gathered}
        \forpmm \\
        MtwoSame_{costs}[i, j] = \begin{cases}
            cost_{MtwobSame} & \text{if } M^2[\rho] = 0\\
            0 & \text{otherwise}
        \end{cases}
    \end{gathered}
\end{equation}
\end{enumerate}
\subsection*{Motion Constraints of the Third Species}
\begin{enumerate}[wide, label=\bfseries 3.P\arabic*]
  \item\label{rule:motion3rd} {\textit{The motion is perceived based on the fourth note.}}

This implies that the costs of the motions and the first species constraints on the motions are deducted from $P[3]$.
\end{enumerate}
%%%%%%%%%%%%%%%%%%%%%


\section*{Constraints of the Fourth Species}

\subsection*{Motion Constraints of the Fourth Species}
\begin{enumerate}[wide, label=\bfseries 4.P\arabic*]
  \item\label{rule:dissolved} {\textit{Dissonant harmonies must be followed by the next lower consonant harmony.}}

\begin{equation}
    \begin{gathered}
        \forall j \in [1, m-1) \quad
        \lnot IsCons[0, j] \implies M_{brut}[0, j] \in \{-1, -2\}
    \end{gathered}
\end{equation}

\item\label{rule:nosecond} {\textit{If the \cf is in the lower part then no second harmony can be preceded by a unison/octave harmony.}}

\begin{equation}
    \begin{gathered}
        \forall j \in [1, m-1)\\
        IsCfB[j+1] \implies H[2, j] \neq 0 \land H[0, j+1] \notin \{1, 2\}
    \end{gathered}
\end{equation}

\end{enumerate}
\subsection*{Harmonic Constraints of the Fourth Species}
\begin{enumerate}[wide, label=\bfseries 4.H\arabic*]
  \item\label{rule:arsiscons} {\textit{Arsis harmonies must be consonant.}}

\begin{equation}
    \begin{gathered}
        \forall j \in [0, m-1) \quad
        H[2, j] \in Cons
    \end{gathered}
    \label{eq:arsiscons}
\end{equation}

\item\label{rule:noseventh} {\textit{If the \cf is in the upper part, then no harmonic seventh interval can occur.}}

\begin{equation}
    \begin{gathered}
        \forall j \in [1, m-1) \quad
        \lnot IsCfB[j] \implies H[0, j] \notin \{10, 11\}
    \end{gathered}
\end{equation}

\item\label{rule:lowpenult4th} \label{rule:uppenult4th} {and \textbf{4.H4} \textit{In the penultimate measure, the harmonic interval of the thesis note must be a major sixth or a minor third depending on the \cf pitch.}}

\begin{equation}
    \begin{gathered}
        H[0, m-2] = \begin{cases}
            9 & \text{if } IsCfB[m-2]\\
            3 & \text{otherwise}
        \end{cases}
    \end{gathered}
\end{equation}
\end{enumerate}
\subsection*{Melodic Constraints of the Fourth Species}
\begin{enumerate}[wide, label=\bfseries 4.M\arabic*]
  \item\label{rule:fullsyncopations} {\textit{Arsis half notes should be the same as their next halves in thesis.}}


\begin{equation}
    \begin{gathered}
        \forall j \in [0, m-1) \quad
        NoSync_{costs} = \begin{cases}
            cost_{NoSync} & \text{if } M[2, j] \neq 0\\
            0 & \text{otherwise}
        \end{cases}
    \end{gathered}
\end{equation}

\item\label{rule:m2same} {\textit{Each arsis note and its two measures further peer are preferred to be different.}}


\begin{equation}
    \begin{gathered}
        \forall j \in [0, m-1)\\
        MtwomSame_{costs} = \begin{cases}
            cost_{MtwomSame} & \text{if } Cp[2, j] = Cp[2, j+2]\\
            0 & \text{otherwise}
        \end{cases}
    \end{gathered}
\end{equation}

\end{enumerate}
%%%%%%%%%%%%%%